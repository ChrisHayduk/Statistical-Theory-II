\documentclass[12pt]{article}
 
\usepackage[margin=1in]{geometry}
\usepackage{amsmath,amsthm,amssymb, mathtools}
\usepackage[T1]{fontenc}
\usepackage{lmodern}
\usepackage{fixltx2e}
\usepackage[shortlabels]{enumitem}
 
\newcommand{\N}{\mathbb{N}}
\newcommand{\R}{\mathbb{R}}
\newcommand{\Z}{\mathbb{Z}}
\newcommand{\Q}{\mathbb{Q}}
 
\newenvironment{theorem}[2][Theorem]{\begin{trivlist}
\item[\hskip \labelsep {\bfseries #1}\hskip \labelsep {\bfseries #2.}]}{\end{trivlist}}
\newenvironment{lemma}[2][Lemma]{\begin{trivlist}
\item[\hskip \labelsep {\bfseries #1}\hskip \labelsep {\bfseries #2.}]}{\end{trivlist}}
\newenvironment{exercise}[2][Exercise]{\begin{trivlist}
\item[\hskip \labelsep {\bfseries #1}\hskip \labelsep {\bfseries #2.}]}{\end{trivlist}}
\newenvironment{problem}[2][Problem]{\begin{trivlist}
\item[\hskip \labelsep {\bfseries #1}\hskip \labelsep {\bfseries #2.}]}{\end{trivlist}}
\newenvironment{question}[2][Question]{\begin{trivlist}
\item[\hskip \labelsep {\bfseries #1}\hskip \labelsep {\bfseries #2.}]}{\end{trivlist}}
\newenvironment{corollary}[2][Corollary]{\begin{trivlist}
\item[\hskip \labelsep {\bfseries #1}\hskip \labelsep {\bfseries #2.}]}{\end{trivlist}}
\newcommand{\textfrac}[2]{\dfrac{\text{#1}}{\text{#2}}}

\begin{document}

\title{Statistical Theory II: Chapter 7 - Estimation}

\author{Chris Hayduk}
\date{\today}

\maketitle

\begin{problem}{8.12}
\end{problem}

\begin{enumerate}[label=\alph*)]
	\item $E(\hat{\theta}) = \overline{Y} = \frac{\theta + (\theta+1)}{2} = \theta + \frac{1}{2}$\\
		  $B(\hat{\theta}) = E(\hat{\theta}) - \theta = \theta + \frac{1}{2} - \theta = \frac{1}{2}$
	\item Let $E(\hat{\theta}) = \overline{Y} - \frac{1}{2} = \theta$\\
		  \\
		  Then,\\
		  $B(\hat{\theta}) = E(\hat{\theta}) - \theta = \theta - \theta = 0$
	\item $MSE(\overline{Y}) = V(\overline{Y}) + [B(\overline{Y})]^2 = \frac{1}{12n} + \frac{1}{2}$
\end{enumerate}

\begin{problem}{8.22}
\end{problem}

Let $b = 2\sigma_{\mu} = 2(\frac{\sigma}{\sqrt{n}}) \approx 2(\frac{\sigma_{\mu}}{\sqrt{2}}) = \frac{5.6}{\sqrt{200}} \approx 0.791960.$\\

Thus, the probability that $\epsilon \leq 0.791960$ is approximately 0.95. As a result, we expect the mean to fall in the range $[6.40804, 7.99196]$ with 95\% certainty.

\begin{problem}{8.46}
\end{problem}

\begin{enumerate}[label=\alph*)]
	\item $m_U(t) = E(e^{tU}) = E(e^{t\frac{2Y}{\theta}}) = m_Y(\frac{2t}{\theta})$\\
	\\
	Since $Y$ is distributed exponentially with mean $\theta$, we know from Example 6.12 that $m_Y(t) = (1 - t\theta)^{-1}$\\
	\\
	Thus,\\
	$m_Y(\frac{2t}{\theta}) = (1 - \frac{2t}{\theta}\theta)^{-1} = (1 - 2t)^{-1}$\\
	\\
	This is the moment generating function for a $\chi^2$-distribution with two degrees of freedom. As a result, $U$ has the same distribution. $U$ is also a pivotal quantity because the distribution does not depend on $\theta$.
	\item From Appendix 3, Table 6 with two degrees of freedom:\\
	$P(0.102587 \leq \frac{2Y}{\theta} \leq 5.99147) = 0.9$\\
	\\
	This yields,\\
	$(\frac{2Y}{5.99147}, \frac{2Y}{0.102587})$\\
	as the 90\% confidence interval for $\theta$.
	\item $\frac{2Y}{5.99147} \approx \frac{Y}{2.996}$ and $\frac{2Y}{0.102587} \approx \frac{Y}{0.051}$\\
	\\
	Thus, the two confidence intervals are equivalent.
\end{enumerate}

\begin{problem}{8.60}
\end{problem}

\begin{enumerate}[label=\alph*)]
	\item From Example 8.6:\\
	\\$\hat{\theta}_L = \hat{\theta} - z_{\alpha/2}\sigma_{\hat{\theta}}$ and $\hat{\theta}_U = \hat{\theta} + z_{\alpha/2}\sigma_{\hat{\theta}}$\\
	\\
	Thus, with $\alpha = 0.01$,\\
	\\
	$\hat{\theta}_L = 98.25 - 2.576(\frac{0.73}{\sqrt{130}}) \approx 98.0851$ and $\hat{\theta}_U = 98.25 + 2.576(\frac{0.73}{\sqrt{130}}) \approx 98.4149$
	\item This confidence interval does not contain the value 98.6 degrees. Thus, we can say with 99\% confidence that 98.6 degrees is not an accurate estimate for the average body temperature of a healthy human.
\end{enumerate}

\begin{problem}{8.102}

\end{problem}
\begin{align*}
		 s^2 &= [\frac{1}{n-1}]\sum\limits_{i=1}^{n} (Y_i - \overline{Y})^2\\
		 &= [\frac{1}{4}]\sum\limits_{i=1}^{5} (Y_i - 57)^2\\
		 &= \frac{289}{2} = 144.5
\end{align*}\\

Thus, with $s^2 = 144.5$, $\frac{\alpha}{2} = 0.005$, and $df = 4$, Table 6, Appendix 3 gives $\chi^2_{0.995} = 0.206990$ and $\chi^2_{0.005} = 14.8602$. Hence, the 90\% confidence interval for $\sigma^2$ is,\\
\\
$(\frac{(4)(144.5)}{14.8602}, \frac{(4)(144.5)}{0.206990}) \approx (38.896, 2792.405)$

\end{document}