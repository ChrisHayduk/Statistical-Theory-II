\documentclass[12pt]{article}
 
\usepackage[margin=1in]{geometry}
\usepackage{amsmath,amsthm,amssymb, mathtools}
\usepackage[T1]{fontenc}
\usepackage{lmodern}
\usepackage{fixltx2e}
\usepackage[shortlabels]{enumitem}
 
\newcommand{\N}{\mathbb{N}}
\newcommand{\R}{\mathbb{R}}
\newcommand{\Z}{\mathbb{Z}}
\newcommand{\Q}{\mathbb{Q}}
 
\newenvironment{theorem}[2][Theorem]{\begin{trivlist}
\item[\hskip \labelsep {\bfseries #1}\hskip \labelsep {\bfseries #2.}]}{\end{trivlist}}
\newenvironment{lemma}[2][Lemma]{\begin{trivlist}
\item[\hskip \labelsep {\bfseries #1}\hskip \labelsep {\bfseries #2.}]}{\end{trivlist}}
\newenvironment{exercise}[2][Exercise]{\begin{trivlist}
\item[\hskip \labelsep {\bfseries #1}\hskip \labelsep {\bfseries #2.}]}{\end{trivlist}}
\newenvironment{problem}[2][Problem]{\begin{trivlist}
\item[\hskip \labelsep {\bfseries #1}\hskip \labelsep {\bfseries #2.}]}{\end{trivlist}}
\newenvironment{question}[2][Question]{\begin{trivlist}
\item[\hskip \labelsep {\bfseries #1}\hskip \labelsep {\bfseries #2.}]}{\end{trivlist}}
\newenvironment{corollary}[2][Corollary]{\begin{trivlist}
\item[\hskip \labelsep {\bfseries #1}\hskip \labelsep {\bfseries #2.}]}{\end{trivlist}}
\newcommand{\textfrac}[2]{\dfrac{\text{#1}}{\text{#2}}}

\begin{document}

\title{Statistical Theory II: Chapter 7 - Sampling Distributions and the Central Limit Theorem}

\author{Chris Hayduk}
\date{\today}

\maketitle

\begin{problem}{7.10}
\end{problem}

\begin{enumerate}[label=\alph*]
	\item\begin{align*}
		 P(|\overline{Y} - \mu| \leq 0.3)
		 &= P[-0.3 \leq (\overline{Y} - \mu) \leq 0.3]\\
		 &= P(-\frac{0.3}{\sigma/\sqrt{n}} \leq \frac{\overline{Y} - \mu}{\sigma/\sqrt{n}} \leq \frac{0.3}{\sigma/\sqrt{n}})
		 \end{align*}\\
		 Since $\frac{\overline{Y} - \mu}{\sigma/\sqrt{n}}$ has a standard normal distribution, we can write:
		 \begin{align*}
		 P(|\overline{Y} - \mu| \leq 0.3) &= P(-\frac{0.3}{2/\sqrt{9}} \leq Z \leq \frac{0.3}{2/\sqrt{9}})\\
		 &= P(-0.45 \leq Z \leq 0.45)
		 \end{align*}\\
		 Using Table 4, Appendix 3 yields:
		 \begin{align*}
		 P(-0.45 \leq Z \leq 0.45) &= 1 - 2P(Z > 0.45)\\&= 1 - 2(0.3264) = 0.3472
		 \end{align*}
		 Thus, the probability is 0.3472 that the sample mean will be within 0.3 ounces of the true mean.
	\item Using varying values of n
	\begin{itemize}
		\item $n = 25$\\
			  \begin{align*}
			  P(|\overline{Y} - \mu| \leq 0.3) &= P(-\frac{0.3}{2/\sqrt{25}} \leq Z \leq \frac{0.3}{2/\sqrt{25}})\\
			  &= P(-0.75 \leq Z \leq 0.75)
			  \end{align*}\\
			  Using Table 4, Appendix 3 yields:
			  \begin{align*}
			  P(-0.75 \leq Z \leq 0.75) &= 1 - 2P(Z > 0.75)\\&= 1 - 2(0.2266) = 0.5468
			  \end{align*}
			  Thus, the probability is 0.5468 that the sample mean will be within 0.3 ounces of the true mean.
		\item $n = 36$\\
			  \begin{align*}
			  P(|\overline{Y} - \mu| \leq 0.3) &= P(-\frac{0.3}{2/\sqrt{36}} \leq Z \leq \frac{0.3}{2/\sqrt{36}})\\
			  &= P(-0.9 \leq Z \leq 0.9)
			  \end{align*}\\
			  Using Table 4, Appendix 3 yields:
			  \begin{align*}
			  P(-0.9 \leq Z \leq 0.9) &= 1 - 2P(Z > 0.9)\\&= 1 - 2(0.1841) = 0.6318
			  \end{align*}
			  Thus, the probability is 0.6318 that the sample mean will be within 0.3 ounces of the true mean.
		\item $n = 49$\\
		    \begin{align*}
			P(|\overline{Y} - \mu| \leq 0.3) &= P(-\frac{0.3}{2/\sqrt{49}} \leq Z \leq \frac{0.3}{2/\sqrt{49}})\\
			&= P(-1.05 \leq Z \leq 1.05)
			\end{align*}\\
			Using Table 4, Appendix 3 yields:
			\begin{align*}
			P(-1.05 \leq Z \leq 1.05) &= 1 - 2P(Z > 1.05)\\&= 1 - 2(0.1469) = 0.7062
			\end{align*}
			Thus, the probability is 0.7062 that the sample mean will be within 0.3 ounces of the true mean.
		\item $n = 64$\\
			\begin{align*}
			P(|\overline{Y} - \mu| \leq 0.3) &= P(-\frac{0.3}{2/\sqrt{64}} \leq Z \leq \frac{0.3}{2/\sqrt{64}})\\
			&= P(-1.2 \leq Z \leq 1.2)
			\end{align*}\\
			Using Table 4, Appendix 3 yields:
			\begin{align*}
			P(-1.2 \leq Z \leq 1.2) &= 1 - 2P(Z > 1.2)\\&= 1 - 2(0.1151) = 0.7698
			\end{align*}
			Thus, the probability is 0.7698 that the sample mean will be within 0.3 ounces of the true mean.
	\end{itemize}
	\item As $n$ increases, the probability that the sample mean is within 0.3 ounces of the true mean increases.
	\item When $\sigma = 1$, the probabilities are much higher. This occurs because there is less variability in the data and, as a result, there is less variability in the sample means.
\end{enumerate}

\begin{problem}{7.48}
\end{problem}

\begin{enumerate}[label=\alph*]
	\item\begin{align*}
	P(|\overline{Y} - \mu| \leq 1)
	&= P[-1 \leq (\overline{Y} - \mu) \leq 1]\\
	&= P(-\frac{1}{\sigma/\sqrt{n}} \leq \frac{\overline{Y} - \mu}{\sigma/\sqrt{n}} \leq \frac{1}{\sigma/\sqrt{n}})
	\end{align*}\\
	Since $\frac{\overline{Y} - \mu}{\sigma/\sqrt{n}}$ has a standard normal distribution, we can write:
	\begin{align*}
	P(|\overline{Y} - \mu| \leq 1) &= P(-\frac{1}{12/\sqrt{35}} \leq Z \leq \frac{1}{12/\sqrt{35}})\\
	&\approx P(-0.49 \leq Z \leq 0.49)
	\end{align*}\\
	Using Table 4, Appendix 3 yields:
	\begin{align*}
	P(-0.49 \leq Z \leq 0.49) &= 1 - 2P(Z > 0.49)\\&= 1 - 2(0.3121) = 0.3758
	\end{align*}
	Thus, the probability is about 0.3758 that the sample mean (26\%) is within 1\% of the mean of the population of estimates of all economists.
	\item No because each measurement is just an estimate of the percent tax saving.
\end{enumerate}

\begin{problem}{7.62}
\end{problem}
Let $Y_i$ denote the service time for the $i$th customer. Then we have:\\
$P(\sum_{i=1}^{100}Y_i > 240) = P(\overline{Y} > \frac{240}{100}) = P(\overline{Y} > 2.40)$

Since the sample size is large, the central limit theorem tells us that $\overline{Y}$ is approximately normally distributed with mean $\mu_{\overline{Y}} = \mu = 2.5$ and variance $\sigma_{\overline{Y}}^2 = \sigma^2/n = 4.0/100$. Therefore, using Table 4, Appendix 3 yields:
\begin{align*}
P(\overline{Y} > 2.40) &= P(\frac{\overline{Y} - 2.50}{\frac{2.0}{\sqrt{100}}} > \frac{2.40 - 2.50}{\frac{2.0}{\sqrt{100}}})\\
&\approx P(Z \leq (2.4 - 2.5)\frac{10}{2}) = P(Z > -\frac{1}{2})\\
&= 1 - P(Z > 0.5) = 1 - 0.3085 = 0.6915
\end{align*}

Thus, the probability is about 0.6915 that it will take more than 4 hours to process the orders of 100 people.

\begin{problem}{7.92}
\end{problem}

\begin{align*}
P(|\overline{X} - \overline{Y}| > 0.6) &= P(|Z| \leq \frac{0.6}{\sqrt{\frac{6.4^2 + 7.2^2}{64}}}\\
	&\approx P(|Z| \leq 0.5) = 0.6170
\end{align*}\\

\begin{problem}{7.96}
\end{problem}

Each $Y_i$ has a beta distribution with $\alpha = 3$ and $\beta = 1$. By Theorem 4.11, $\mu = \frac{\alpha}{\alpha + \beta} = \frac{3}{4}$ and $\sigma^2 = \frac{\alpha\beta}{(\alpha + \beta)^2(\alpha + \beta + 1)} = \frac{3}{80}$.\\
Thus,\\
\begin{align*}
P(\overline{Y} > 0.7) &\approx P(Z > \frac{0.7 - 0.75}{\sqrt{\frac{0.0375}{40}}})\\
	&= P(Z > -1.63) = 1 - P(Z > 1.63) = 1 - 0.0516 = 0.9484
\end{align*}\\
Thus, the probability that the ore will be rejected is about 0.9484.
\end{document}